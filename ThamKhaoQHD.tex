\title{Bài thi cuối kỳ Thực hành Toán tổ hợp}
\author{Hùng Ngọc Phát -- 19120615}

%%%%%%%%%%%%%%%%%%%%%% PACKAGE INCLUSIONS %%%%%%%%%%%%%%%%%%%%%% 
\documentclass[12pt]{article}
\usepackage{extsizes}
\usepackage{mathrsfs}
\usepackage[T5]{fontenc}
\usepackage[dvipsnames]{xcolor}
\usepackage{csquotes}
\usepackage{caption}
\usepackage[vietnamese,english]{babel}
\usepackage{amsmath}
\usepackage{amssymb}
\usepackage{centernot}
\usepackage[outputdir=build,cache=false]{minted}
% Xoá đoạn "[outputdir=build,cache=false]" ở dòng trên nếu compile trên Overleaf
\usepackage{float}
\usepackage{graphicx}
\usepackage[colorinlistoftodos]{todonotes}
\usepackage{listings}
\usepackage[unicode]{hyperref}
\usepackage{enumitem}
\usepackage{subfigure}
\usepackage{fancyhdr}
\usepackage{subfiles}
\usepackage{wrapfig}
\usepackage{forest}
\usetikzlibrary{arrows.meta,
                matrix,
                chains,
                positioning,
                quotes,
                shapes.geometric}

\usepackage{geometry}

%%%%%%%%%%%%%%%%%%%%%% DOCUMENT FORMATTING %%%%%%%%%%%%%%%%%%%%%% 
\geometry{
    a4paper,
    total={170mm,250mm},
    left=20mm,
    top=30mm,
 }
\hypersetup{
    colorlinks=true,
    linkcolor=blue,
    filecolor=magenta,      
    urlcolor=blue,
    citecolor=blue
}

\pagestyle{fancy}
\fancyhf{}
\rhead{CLB Học thuật NES}
\lhead{Tài liệu ôn thi Kỹ thuật Lập trình}
\rfoot{Trang \thepage}

\setlength{\parindent}{0pt}
\setlength{\parskip}{0.3em}
\setlength{\headheight}{15pt}

\AtBeginEnvironment{minted}{
    \renewcommand{\fcolorbox}[4][]{#4}}       

%%%%%%%%%%%%%%%%%%%%%% MACRO DEFINITIONS %%%%%%%%%%%%%%%%%%%%%% 
\newcommand{\nocontentsline}[3]{}
\newcommand{\tocless}[2]{\bgroup\let\addcontentsline=\nocontentsline#1{#2}\egroup}

\newcommand{\cd}[1]{\texttt{#1}}

\renewenvironment{figure*}
{\begin{figure}[H]\center}
{\end{figure}}

\newcommand{\cdh}[1]{\textcolor{red}{\cd{#1}}}

\renewcommand{\a}[1]{\ensuremath{a_{#1}}}
\newcommand{\Ra}{\ensuremath{\Rightarrow} }
\newcommand{\ssum}[3]{\sum_{#1}^{#2}{#3}}
\newcommand{\deriv}[1]{\ensuremath{\frac{d}{dx}#1}}

\begin{document}
\begin{center}
    \textbf{\Large Tài liệu tham khảo Quy hoạch động}
\end{center}

Đây là một bài trong đề thi môn \textit{Thực hành Toán tổ hợp} của trường mình năm 2021 (đây là môn bắt buộc của CNTT). Nó có liên quan đến đệ quy và quy hoạch động nên tụi mình muốn gửi để các bạn tham khảo. Bài giải này là bài giải \textit{thật} do một thành viên K19 của CLB đóng góp. Cuối cùng mình có đính kèm 3 cách giải bài này theo quy hoạch động bằng Python và 1 cách theo C++. \bigskip 

CLB và các cá nhân liên quan không sở hữu đề bài này.

\tableofcontents

\section{Đề}

Tìm thuật toán để giải quyết bài toán sau đây, gọi là
\textit{Bài toán đường đi tránh vũng nước}: \bigskip

Xét góc phần tư thứ $1$ của hệ trục tọa độ Oxy với các lưới nguyên (giống như giấy tập có ô), có một vũng nước $S$ (là một tập hợp các điểm nguyên nằm trong góc phần tư thứ $1$). \bigskip

a. Tìm số đường đi từ O đến A tránh vũng nước S biết mỗi
lần đi, chỉ được lên trên 1 đơn vị hoặc qua phải 1 đơn vị.\\
b. Có nhận xét gì về số đường đi như vậy nếu vũng nước S
không tồn tại? \bigskip

\begin{figure}[H]
    \centering
    \includegraphics[width=0.43\textwidth]{image/qhd.png}
\end{figure}

\renewcommand{\a}[2]{\ensuremath{\a_{#1,\ #2}}}
\newcommand{\s}{\ensuremath{\mathcal S} }

\section{Giải}
Đây là phần ``suy luận'' ra công thức truy hồi. Nếu các bạn không muốn đọc có thể nhảy xuống phần \ref{code} (Code) ở dưới. Lưu ý là trong phần suy luận này mình trình bày theo kiểu bottom-up.

\subsection{Câu a}
\subsubsection{Mô phỏng ý tưởng ban đầu}
Em xin mô hình hoá bài toán đã cho thành một bài toán quy hoạch động. Để dễ mô hình hoá, em xin biểu diễn lại hệ trục toạ độ đã cho bằng một ma trận có kích thước $(n+1)\times (m+1)$ với $m, n$ là toạ độ của điểm $A(m,n)$ đã cho trong đề bài. Khi đó bài toán có thể phát biểu lại như sau:\\
\textit{Cho một lưới hình chữ nhật có kích thước $(n+1)\times(m+1)$, có bao nhiêu cách để đi từ điểm có toạ độ $(0, 0)$ đến điểm có toạ độ $(n, m)$ mà chỉ đi lên trên hoặc đi qua phải, mỗi lần đi 1 đơn vị?}\bigskip

Giả sử ta quy ước ô nằm ở góc dưới cùng bên trái là $(0,0)$ và trên cùng bên phải là $(n,m)$.\par

\tikzstyle{matnode} = [inner sep=0pt,text width=1cm,align=center,minimum height=1cm]
\newcommand{\empt}{\hphantom{1pt}}

\begin{wrapfigure}{L}{0.34\textwidth}
    \begin{tikzpicture}
        \draw[step=1cm,color=gray] (-2,-2) grid (2,2);
        \matrix[matrix of nodes,nodes={matnode}]{
            \empt & \empt & \empt & (n,m)\\
            \empt & \empt & \empt & \empt\\
            \empt & \empt & \empt & \empt\\
            (0,0) & \empt & \empt & \empt\\
        };

        \draw[latex-latex] (-2,-2.5) -- ++(4,0) node[midway,fill=white] {m+1};
        \draw[latex-latex] (3,-2) -- ++(0,4) node[midway,fill=white] {n+1};
        \end{tikzpicture}
    \vspace*{-1cm}
\end{wrapfigure}

Ta gọi cách để đi từ ô có toạ độ $(0, 0)$ (ô bắt đầu) để đi đến một ô có toạ độ $(i,j)$ là $a(i, j)$ với $i, j$ lần lượt là chỉ số hàng và chỉ số cột của ô đó.

Vì từ ô bắt đầu ta chỉ có thể đi sang trái hoặc sang phải, nên ta nhận thấy rằng để đi từ ô bắt đầu tới ô có toạ độ $(i,j)$ bất kỳ thì ta chỉ có 2 cách đi: tới từ ô ở ngay dưới bên dưới $(i-1,j)$ hoặc từ ô ở ngay bên phải $(i,j-1)$. \par Do đó, cách để đi đến một ô có vị trí $(i, j)$ bất kỳ là:
$$
a(i, j) = a(i-1, j) + a(i, j-1)
$$

\par Ta nhận thấy rằng để đi từ ô có vị trí $(0, 0)$ đến ô có vị trí $(0, 0)$ thì ta có duy nhất 1 cách đi là không làm gì cả, nên $a(0,0) = 1$. \\ Để dễ tính toán, ta quy ước không có cách đi nào đến những ô có toạ độ âm, nên $a(i, j) = 0$ nếu $i, j < 0$. 

Như vậy, ta có thể rút ra thuật toán cho bài toán này là (\textit{hiện tại ta chưa xét các ``vũng nước''}):
\begin{itemize}
    \item Bắt đầu tại ô $(0, 0)$. Lần lượt đi từ trái qua phải, từ dưới lên trên và cập nhật giá trị cho các ô đã đi qua theo công thức:
    $$
    a(i,j) = \begin{cases}
        1 &, (i, j) = (0, 0) \\
        0 &, i < 0 \vee j < 0 \\
        a(i, j-1) + a(i-1,j) &, \text{các TH còn lại}    
    \end{cases}
    $$
    \item Lặp lại bước trên và thoát khỏi thuật toán sau khi gán xong giá trị cho ô $(n, m)$.
    \item Kết quả của bài toán (số đường đi từ O tới A) là giá trị tại ô $(n,m)$.
\end{itemize}

Giả sử khi $m=3, n=4$ thì ma trận thu được sẽ là:
\begin{figure}[H]
    \centering
    \begin{tikzpicture}
        \matrix[matrix of nodes,nodes={draw=gray, anchor=center, minimum size=0.8cm}, column sep=-\pgflinewidth, row sep=-\pgflinewidth] (A) {
            1 & 5 & 15 & \textcolor{blue}{35}  \\
            1 & 4 & 10 & 20\\
            1 & 3 & 6 & 10\\
            1 & 2 & 3 & 4 \\
            1 & 1 & 1 & 1\\};
        \end{tikzpicture}
\end{figure}
$\Ra$ Có 35 cách đi từ $O$ tới $A(3,4)$. Ta có thể kiểm chứng bằng công thức chứng minh được ở câu b: $C_{3+4}^{3} = 35$.

\subsubsection{Thuật toán cho trường hợp có vũng nước}
Gọi \s là tập hợp những điểm trên hệ trục toạ độ bị ``chìm'' trong vũng nước.\par
Ta dễ dàng nhận thấy ở những ô chìm trong vũng nước thì không có cách nào để đi đến đó, nên $a(i,j)=0$ nếu $(i,j) \in \s$.\par 
Thuật toán cho trường hợp này tương tự trường hợp trên, chỉ có điều lần này ta sẽ xét thêm các ô đó có nằm trong vũng nước hay không. Cụ thể như sau:
\begin{itemize}
    \item Bắt đầu tại ô $(0, 0)$. Lần lượt đi từ trái qua phải, từ dưới lên trên và cập nhật giá trị cho các ô đã đi qua theo công thức:
    $$
    a(i,j) = \begin{cases}
        1 &, (i, j) = (0, 0) \\
        0 &, i < 0 \vee j < 0 \vee (i,j) \in \s \\
        a(i, j-1) + a(i-1,j) &, \text{các TH còn lại}    
    \end{cases}
    $$
    \item Lặp lại bước trên và thoát khỏi thuật toán sau khi gán xong giá trị cho ô $(n, m)$.
    \item Kết quả của bài toán là giá trị tại ô $(n,m)$.
\end{itemize}

\begin{wrapfigure}{L}{0.4\textwidth}
    \includegraphics[width=0.39\textwidth]{image/Bai8.png}
    \vspace*{-4cm}
\end{wrapfigure}

\textbf{Ví dụ:} xét điểm $A(5, 6)$ và vũng nước như hình bên.

Ta nhận thấy rằng các điểm trên hệ trục toạ độ bị ``chìm'' trong vũng nước là\\ $\left\{(2, 2), (2, 3), (2, 4), (3, 3), (3, 4)\right\}$.\\ Vậy, $\s = \left\{(2, 2), (3, 2), (4, 2), (3, 3), (4, 3)\right\}$ (do ký hiệu toạ độ trong hệ trục $Oxy$ và ma trận bị ngược nhau).\par Khi thực hiện thuật toán, các ô mang toạ độ này sẽ chứa giá trị $0$. \par 
Vậy ta có thể thành lập một ma trận $7\times 6$ như sau:.\\[2cm]

\begin{figure}[H]
    \newcommand{\0}{\textcolor{red}{0}}
    \centering
    \begin{tikzpicture}
        \matrix[matrix of nodes,nodes={draw=gray, anchor=center, minimum size=1cm}, column sep=-\pgflinewidth, row sep=-\pgflinewidth] (A) {
            1 & 7 & 13 & 19 & 34 & \textcolor{blue}{82}\\
            1 & 6 & 6 & 6 & 15 & 48\\
            1 & 5 & \0 & \0 & 9 & 33\\
            1 & 4 & \0 & \0 & 9 & 24\\
            1 & 3 & \0 & 4 & 9 & 15\\
            1 & 2 & 3 & 4 & 5 & 6\\
            1 & 1 & 1 & 1 & 1 & 1\\};
        \end{tikzpicture}
\end{figure}

Vậy đáp số của ví dụ này là 82. \bigskip 

\textit{(còn nữa ở trang sau)}
\pagebreak

\subsection{Câu b}
\begin{wrapfigure}{L}{0.4\textwidth}
    \includegraphics[width=0.39\textwidth]{image/Bai8b.png}
    \caption*{Đường đi ứng với chuỗi \cd{URRURUUURRU}}
    \vspace*{-2cm}
\end{wrapfigure}
Nếu không tồn tại vũng nước, ta nhận thấy rằng để đi từ gốc toạ độ $O$ đến $A(m, n)$ luôn luôn tốn $m+n$ bước. \par Ví dụ như xét điểm $A(5, 6)$ như hình bên, ta luôn luôn tốn $5+6=11$ bước để đi từ $O$ đến $A$.\par 
Nếu ta ký hiệu một bước đi lên là ký tự \cd U và một bước sang phải là ký tự \cd R, một đường đi sẽ tương ứng với một chuỗi các kí tự \cd{U, R}. Ta nhận thấy nếu ta đổi chỗ các kí tự \cd U hoặc các kí tự \cd R cho nhau thì chuỗi trên vẫn không thay đổi, nên bài toán sẽ trở thành: có bao nhiêu cách chọn $m+n$ kí tự từ $m$ kí tự \cd R và $n$ kí tự \cd U? \bigskip

Ta có thể dễ dàng suy ra số cách chọn từ kiến thức về tổ hợp:
$$
{m+n \choose m} = {m + n \choose n}
$$
\\[0.5cm]


Lưu ý về ký hiệu: 
$$
{n \choose k} = C_{n}^{k}
$$\bigskip 

\textit{(còn nữa ở trang sau)}
\pagebreak

\section{Code} \label{code}
Các bạn có thể copy nội dung của các đoạn code trong phần này trong thư mục \verb#answer-sources# ở link sau: \href{https://github.com/hungngocphat01/nes-ktlt-2021}{Github: hungngocphat01/nes-ktlt-2021}. \bigskip 

\paragraph{Tại sao lại Python?} Vì việc biểu diễn thuật toán bằng Python rất tự nhiên (giống y như khi bạn nói tiếng Anh). C++ viết thuật toán rất mệt (mình không muốn sử dụng cấp phát động, đặc biệt với những kiểu phức tạp). Nhưng ở cuối file mình cũng viết mẫu 1 bài nhỏ bằng C++. Phần này mình thêm vào sau chứ môn \textit{Toán tổ hợp} này thì không có code. \bigskip 

\inputminted[linenos,breaklines]{python}{answer_sources/QHD_ThamKhao.py}

Sau khi chạy, các bạn sẽ thu được kết quả như sau:
\begin{figure}[H]
    \centering 
    \includegraphics[width=\textwidth]{image/qhd_run.png}
\end{figure}
Trong đó 2 cái đầu tiên có quy hoạch động chạy rất nhanh (dưới 0.1s là đã giải xong). Cái cuối cùng giải rất lâu, gần như \textit{không thể giải ra}. \bigskip 

Dưới đây là cài đặt lại hàm cuối cùng (bottom up) bằng C++ (cái này dễ cấp phát nên mình code lại cho các bạn dễ hiểu):
\inputminted[linenos,breaklines]{cpp}{answer_sources/QHD_ThamKhao2.cpp}
Kết quả thu được sẽ hoàn toàn tương tự.
\end{document}