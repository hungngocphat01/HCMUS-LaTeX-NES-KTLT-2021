%%%%%%%%%%%%%%%%%%%%%%%%%%%%%%%%%%%%%%%%%
% University Assignment Title Page 
% LaTeX Template
% Version 1.0 (27/12/12)
%
% This template has been downloaded from:
% http://www.LaTeXTemplates.com
%
% Original author:
% WikiBooks (http://en.wikibooks.org/wiki/LaTeX/Title_Creation)
%
% License:
% CC BY-NC-SA 3.0 (http://creativecommons.org/licenses/by-nc-sa/3.0/)
%
%%%%%%%%%%%%%%%%%%%%%%%%%%%%%%%%%%%%%%%%%

\title{Ôn tập cuối kỳ môn Kỹ thuật lập trình}
%%%%%%%%%%%%%%%%%%%%%% PACKAGE INCLUSIONS %%%%%%%%%%%%%%%%%%%%%% 
\documentclass[12pt]{article}
\usepackage[T5]{fontenc}
\usepackage[utf8]{inputenc}
\usepackage{csquotes}
\usepackage[vietnamese,english]{babel}
\usepackage{amsmath}
\usepackage[outputdir=build,cache=false]{minted}
\usepackage{float}
\usepackage{graphicx}
\usepackage[colorinlistoftodos]{todonotes}
\usepackage{listings}
\usepackage[unicode]{hyperref}
\usepackage{enumitem}
\usepackage{fancyhdr}
\usepackage{subfiles}
\usepackage{background}
\backgroundsetup{angle=0,contents=\includegraphics{image/neslogo.jpg},opacity=0.07,scale=0.3}
\usepackage{geometry}

%%%%%%%%%%%%%%%%%%%%%% DOCUMENT FORMATTING %%%%%%%%%%%%%%%%%%%%%% 
\geometry{
    a4paper,
    total={170mm,250mm},
    left=20mm,
    top=30mm,
 }
\hypersetup{
    colorlinks=true,
    linkcolor=blue,
    filecolor=magenta,      
    urlcolor=blue,
    citecolor=blue
}

\pagestyle{fancy}
\fancyhf{}
\rhead{Câu lạc bộ Học thuật NES}
\lhead{Tài liệu ôn tập cuối kỳ Kỹ thuật lập trình}
\rfoot{Trang \thepage}

\setlength{\parindent}{0pt}
\setlength{\parskip}{0.3em}
\setlength{\headheight}{15pt}

\AtBeginEnvironment{minted}{
    \renewcommand{\fcolorbox}[4][]{#4}}
%%%%%%%%%%%%%%%%%%%%%% MACRO DEFINITIONS %%%%%%%%%%%%%%%%%%%%%% 
\newcommand{\nocontentsline}[3]{}
\newcommand{\tocless}[2]{\bgroup\let\addcontentsline=\nocontentsline#1{#2}\egroup}

\newcommand{\code}[1]{\texttt{#1}}

\renewenvironment{figure*}
{\begin{figure}[H]\center}
{\end{figure}}

\begin{document}
%%%%%%%%%%%%%%%%%%%%%% TITLE PAGE %%%%%%%%%%%%%%%%%%%%%% 
\begin{titlepage}
% \NoBgThispage

\newcommand{\HRule}{\rule{\linewidth}{0.5mm}} % Defines a new command for the horizontal lines, change thickness here

\center % Center everything on the page
\vspace*{\fill}
 
\textsc{\LARGE Đại học Khoa học Tự nhiên}\\[0.2cm]
\textsc{\large Đại học Quốc gia TP. HCM }\\[1.5cm] 
\textsc{\Large Câu lạc bộ Học thuật NES}\\[0.2cm] 
\textsc{\large Mảng Điện tử - Kỹ thuật }\\[0.5cm]
\HRule \\[0.4cm]
{ \huge \bfseries Tài liệu ôn thi cuối kỳ môn\break 
Kỹ thuật lập trình}\\[0.4cm] % Title of your document
\HRule \\[1.5cm]
\LARGE Bản thảo số 1 ngày 22/06/2021 \\
\LARGE Lưu hành nội bộ
~
\begin{minipage}{1\textwidth}
\begin{center}
    \LARGE Học kỳ 2, năm học 2020 - 2021
\end{center}
\end{minipage}\\[2cm]
\includegraphics[width=10em]{image/neslogo.jpg}
\vspace*{\fill} % Fill the rest of the page with whitespace
\end{titlepage}
%%%%%%%%%%%%%%%%%%%%%% DOCUMENT CONTENT %%%%%%%%%%%%%%%%%%%%%% 
\renewcommand*\contentsname{Mục lục}
\setcounter{tocdepth}{2}
\tableofcontents
\pagebreak

\section{Đề bài}
\subfile{section_question}

\section{Đáp án và giải thích}
\subfile{section_answer}

% \bibliographystyle{IEEEtran}
% \bibliography{bib}
\end{document}

%%%%%%%%%%%%%%%%%%%%%%%%%%%%%%%%%%%%%%%%%%%%%%%%%%%%
% Comments can be added to the margins of the document using the \todo{Here's a comment in the margin!} todo command, as shown in the example on the right. You can also add inline comments too:

% \todo[inline, color=green!40]{This is an inline comment.}



% \subsection{Tables and Figures}

% Use the table and tabular commands for basic tables --- see Table~\ref{tab:widgets}, for example. You can upload a figure (JPEG, PNG or PDF) using the files menu. To include it in your document, use the includegraphics command as in the code for Figure~\ref{fig:frog} below.

% % % Commands to include a figure:
% % \begin{figure}
% % \centering
% % \includegraphics[width=0.5\textwidth]{frog.jpg}
% % \caption{\label{fig:frog}This is a figure caption.}
% % \end{figure}

% % \begin{table}
% % \centering
% % \begin{tabular}{l|r}
% % Item & Quantity \\\hline
% % Widgets & 42 \\
% % Gadgets & 13
% % \end{tabular}
% % \caption{\label{tab:widgets}An example table.}
% % \end{table}

% \subsection{Mathematics}

% \LaTeX{} is great at typesetting mathematics. Let $X_1, X_2, \ldots, X_n$ be a sequence of independent and identically distributed random variables with $\text{E}[X_i] = \mu$ and $\text{Var}[X_i] = \sigma^2 < \infty$, and let
% $$S_n = \frac{X_1 + X_2 + \cdots + X_n}{n}
%       = \frac{1}{n}\sum_{i}^{n} X_i$$
% denote their mean. Then as $n$ approaches infinity, the random variables $\sqrt{n}(S_n - \mu)$ converge in distribution to a normal $\mathcal{N}(0, \sigma^2)$.

% \subsection{Lists}

% You can make lists with automatic numbering \dots

% \begin{enumerate}
% \item Like this,
% \item and like this.
% \end{enumerate}
% \dots or bullet points \dots
% \begin{itemize}
% \item Like this,
% \item and like this.
% \end{itemize}

% We hope you find write\LaTeX\ useful, and please let us know if you have any feedback using the help menu above.

